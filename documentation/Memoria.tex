\documentclass[14pt]{extarticle}

% Esto es para poder escribir acentos directamente:
\usepackage[utf8]{inputenc}
% Esto es para que el LaTeX sepa que el texto est en espaol:
\usepackage[spanish, activeacute]{babel}

% Paquetes de la AMS:
\usepackage{amsmath, amsthm, amsfonts}

\usepackage{extsizes}

\usepackage{pdflscape}
\usepackage{fontawesome}

\usepackage{graphicx}
\graphicspath{ {./images/} }

\usepackage[margin=1in]{geometry}

\usepackage{listings}
\usepackage{natbib}
\usepackage{url}
\usepackage{hyperref} % For hyperlinks in the PDF
\usepackage{multirow}
\usepackage{xifthen}
\usepackage{tabularx}

% Teoremas
%--------------------------------------------------------------------------
\newtheorem{thm}{Teorema}[section]
\newtheorem{cor}[thm]{Corolario}
\newtheorem{lem}[thm]{Lema}
\newtheorem{prop}[thm]{Proposicin}
\theoremstyle{definition}
\newtheorem{defn}[thm]{Definicin}
\theoremstyle{remark}
\newtheorem{rem}[thm]{Observacin}

% Atajos.
% Se pueden definir comandos nuevos para acortar cosas que se usan
% frecuentemente. Como ejemplo, aqu se definen la R y la Z dobles que
% suelen representar a los conjuntos de nmeros reales y enteros.
%--------------------------------------------------------------------------

\def\RR{\mathbb{R}}
\def\ZZ{\mathbb{Z}}

% De la misma forma se pueden definir comandos con argumentos. Por
% ejemplo, aqu definimos un comando para escribir el valor absoluto
% de algo ms fcilmente.
%--------------------------------------------------------------------------
\newcommand{\abs}[1]{\left\vert#1\right\vert}

\newcommand{\newtableitem}[1] {
	\multicolumn{4}{|l|}{· #1} \\
}

\newcommand{\newtableline}[1] { %55 caracteres
	\multicolumn{4}{|l|}{#1} \\
}

\newcommand{\userstory}[9]{
	\begin{center}
	\resizebox{15cm}{!} {
	\begin{tabular}{|c|c|c|c|}
    	\hline 
		\multicolumn{4}{|c|}{\textbf{#1}} \\
		\multicolumn{4}{|c|}{} \\
		\hline	
		\textbf{Identificador:} #2 & \textbf{Prioridad:} #3 & \textbf{Iteración:} #4 & \textbf{Puntos de historia:} #5 \\ 
		\hline 
		\multicolumn{4}{|l|}{\textbf{Descripción:}} \\
		#6
		\hline 
		\multicolumn{4}{|l|}{\textbf{Tareas:}} \\
		#7
		\hline 
		\multicolumn{4}{|l|}{\textbf{Pruebas de aceptación:}} \\
		#8
		\hline 
		\multicolumn{4}{|l|}{\textbf{Observaciones:}} \\
		#9
		\hline 
	\end{tabular}
	}
	\end{center}
}

\newcommand{\sprinttask}[2] {
	\multicolumn{1}{|>{\hsize=\dimexpr1\hsize+1\tabcolsep+\arrayrulewidth\relax}X|}{#1} & \multicolumn{3}{|>{\hsize=\dimexpr3\hsize+3\tabcolsep+\arrayrulewidth\relax}X|}{#2}\\
}

\newcommand{\sprint}[4]{
	\begin{center}
	\begin{tabularx}{\textwidth}{|X|X|X|X|}
    	\hline 
		\multicolumn{2}{|>{\hsize=\dimexpr2\hsize+2\tabcolsep+\arrayrulewidth\relax}X|}{\textbf{Sprint:} #1} &
		\multicolumn{2}{|>{\hsize=\dimexpr2\hsize+2\tabcolsep+\arrayrulewidth\relax}X|}{\textbf{Inicio:} #2} \\
		\hline 
		Versión & \multicolumn{3}{|>{\hsize=\dimexpr3\hsize+3\tabcolsep+\arrayrulewidth\relax}X|}{#4} \\
		\hline 
		Identificador & \multicolumn{3}{|>{\hsize=\dimexpr3\hsize+3\tabcolsep+\arrayrulewidth\relax}X|}{Historia de usuario} \\
		\hline
		#3
		\hline 
	\end{tabularx}
	\end{center}
}

% Operadores.
% Los operadores nuevos deben definirse como tales para que aparezcan
% correctamente. Como ejemplo definimos en jacobiano:
%--------------------------------------------------------------------------
\DeclareMathOperator{\Jac}{Jac}

%--------------------------------------------------------------------------
%\title{Trabajo Fin de Máster \\  TweetSC: Corrector de texto para twitter.}
%\author{Javier Moreno}
\begin{document}
%\maketitle
\begin{titlepage}

\begin{center}
\vspace*{-1in}
\begin{figure}[htb]
\begin{center}
\includegraphics[scale=0.8]{logoUPM.jpg}
\end{center}
\end{figure}
ESCUELA TÉCNICA SUPERIOR DE INGENIEROS INFORMÁTICOS\\
\vspace*{0.15in}
MÁSTER UNIVERSITARIO EN INTELIGENCIA ARTIFICIAL \\
\vspace*{0.6in}
\begin{large}
TRABAJO DE FIN DE MÁSTER:\\
\end{large}
\vspace*{0.2in}
\begin{Large}
\textbf{TWEETSC: CORRECTOR DE TEXTO PARA TWITTER} \\
\end{Large}
\vspace*{0.3in}
\begin{large}
JAVIER MORENO VEGA\\
\end{large}
\vspace*{0.3in}
\rule{80mm}{0.1mm}\\
\vspace*{0.1in}
\begin{large}
TUTOR DE PROYECTO: \\
OSCAR CORCHO GARCÍA \\
\vspace*{0.3in}
CO-TUTOR DE PROYECTO: \\
VÍCTOR RODRÍGUEZ DONCEL \\
\end{large}
\vspace*{2in}
\url{http://tweetsc.github.io}
\vspace*{0.2in}\\
\today
\end{center}
\end{titlepage}
\newpage
\tableofcontents % índice de contenidos
\cleardoublepage
\section{Introducción}\label{sec:introduccion}
Prueba referencia \citep{wiki:test}
\subsection{Motivación}\label{sec:motivacion}
\subsection{Objetivos}\label{sec:objetivos}
\subsection{Resumen del documento}\label{sec:resumen}
\section{Estado del arte}\label{sec:estadodelarte}
\section{Análisis y diseño}\label{sec:analisisydiseno}
\subsection{Metodología de desarrollo}\label{sec:metodologiadedesarrollo}
\subsection{Análisis de requisitos}\label{sec:analisisderequisitos}
\subsection{Herramientas utilizadas}
\section{Apéndices}\label{sec:apendices}
\subsection{Apéndice A: Bibliografía}\label{sec:bibliografia}
\bibliographystyle{plain}
\bibliography{References}
\newpage
\subsection{Apéndice B: Glosario de Términos}\label{sec:glosariodeterminos}
\end{document}
